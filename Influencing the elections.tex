% !TeX program = pdflatex
% !TeX encoding = UTF-8
% !TeX spellcheck = en_US

\documentclass[11pt]{article}
\usepackage{fullpage}
\usepackage{amsmath}
\usepackage{graphicx,amsfonts,amsmath,amssymb,epsfig,color,multicol,pstricks}
\input{pream}

\title{Influencing elections}
\date{}

\begin{document}
\maketitle
\section{Introduction}
We are going to study how people changing their votes from current day to the election day affects the outcome of the election. 

In this problem(paper)? we study how the chances of a candidate winning are affected by people changing their vote with a certain probability. %We look at the probability of the candidate winning  
Every voting system is nothing but a boolean function 
 $f:\{0,1\}^{n} \rightarrow \{0,1\}$ e.g Majority.
we will start by considering Majority function and later extend it to other functions.\\
\noindent
 $X =$  people's current preferences \\ \noindent
 $Y =$  people's preferences on election day

\subsection{Unbiased Noise}

$Y$ could be defined in multiple ways, but the simplest and most intuitive definition of $Y$ is the following:
% First possible definition of  \\
\begin{equation*}
 y_i  =\begin{cases}
 x_i , & \text{w.p  $1-\rho$ }.\\
 1-x_i , & \text{w.p  $\rho$ }.
\end{cases}
\end{equation*}

%hi hello

 \BD $T_{\rho}(f,x)$ = $\underset{y \sim N_{\rho}(x)}{\mathbb{E}}[f(y)]$ 
 \ED 
 
\noindent $T_{\rho}(f,x)$ is the probability candidate $1$ wins. By candidate 1 winning we mean the output of the boolean function is 1. The main question we consider is how many bits of x must I query to estimate $T_{\rho}(f,x)$ up to some error $\pm \gamma $ with confidence $1-\delta$? 



\subsubsection{Observations}
\BP With $\rho = 0$, we might need $\Omega(n)$ queries when $\|X\|$ $\simeq$ $\frac{n}{2}$  (for f = Majority).\EP
\BPF
As $\rho=0$, $f(y) = f(x) = constant$,  $T_{\rho}(f,x) = f(x)$ is either $0$ or $1$. In this case talking about $\epsilon$ doesn't make much sense as the error is either $0$ or $1$. So the question reduces to estimating the value of $f(x)$ correctly with probability $1-\delta$. Consider the case where input has $n+1$ bits, out of these $\frac{n}{2}$ are $0's$, $\frac{n}{2}$ are $1's$ and one of them is a qubit with state $\frac{1}{\sqrt{2}}(|0\rangle + |1\rangle)$. The answer to this function depends on which state the qubit collapses to, So if the person does not query the qubit any estimate of $f(x)$ he/she gives is correct only with probability $\frac{1}{2}$. The person should query the qubit with probability at least $1-2\delta$ otherwise the probability that he will give a wrong estimate will be greater than $\delta$. That means he has to make at least $(1-2\delta)n$ queries as he doesen't know which of the bits is the qubit. Hence number of queries required is $\Omega(n)$.
 \EPF

\BP For $\rho= \frac{1}{2}$, we need 0 queries. \EP
\BPF
Since $\rho = 0$, each bit will be $0$ with probability $\frac{1}{2}$ and $1$ with probability $\frac{1}{2}$. So $f(y)=0$ with probability $\frac{1}{2}$ and $f(y)=1$ with probability $\frac{1}{2}$. Hence $T_{\rho}(f,x)= \frac{1}{2}$ and we don't need to make any queries. 
\EPF \\

 \noindent
As for $\rho = 0$ it takes $\Omega (n)$ queries and for $\rho = \frac{1}{2}$ it takes 0. It will be interesting to see what will be the number of queries need for $\rho = \frac{1}{4}$. \\\\
\noindent
An alternate definition of y could be:

\begin{equation}
y_i  =\begin{cases}
1 , & \text{if $x_i = 1$ }.\\
1 , & \text{w.p $\rho$ if $x_i = 0$ }.\\
0 , & \text{w.p $1- \rho$ if $x_i = 0$ }.
\end{cases}
\end{equation}

\subsubsection{Unbiased Noise Operator on Majority function}

We will be looking at $\rho= \omega (\frac{1}{n})$ and $\rho \leq \frac{1}{2}$.

\BT
We can estimate $T_{\rho} \otimes Maj(X)$ to accuracy $\pm \gamma$ with $q=??$ queries. 
\ET

\BL
To estimate $\frac{\|X\|}{n}$ up to $\pm \epsilon$ with confidence 1-$\delta$, $O(\frac{1}{\epsilon^2}log\frac{1}{\delta})$ queries will suffice. 
\EL
\BPF
This can be proved using the Chernoff bound 

$$Pr[a \geq (1+\epsilon)\mu b] \leq e^{\frac{- \epsilon^2 \mu b}{3}}, Pr[a \leq (1-\epsilon)\mu b] \leq e^{\frac{- \epsilon^2 \mu b}{3}}$$
\noindent
Where $a$ = Number of $1's$ encountered, $b$ = Number of bits queried and $\mu = \frac{\|X\|}{n}$. By upper bounding the sum of these probabilities by $\delta$ we get b = $O(\frac{1}{\epsilon^2}log\frac{1}{\delta})$.
\EPF

 One approach could be to relate the error of  $T_\rho \otimes Maj(X) $ and the error in fraction of 1's. That way we will be able to make an estimate on how many queries we need to make to get a good estimate on $T_{\rho} \otimes Maj(X) $, as we already know a relation between the error in fraction of 1's and number of queries. \\


\BCM

Let $A \sim Bin(\alpha n, \rho)$, $B \sim Bin((1-\alpha)n, \rho)$, $\alpha = \frac{\|x\|}{n}$ and $y \sim N_{\rho}(x)$. Then $$T_\rho(Maj, x) = \Pr[B - A \geq (\frac{1}{2} - \alpha)n].$$

\ECM

\BPF

Let $$A = \big| \{ i \in [n]: x_i = 1, y_i = 0\} \big|, B = \big| \{ i \in [n]: x_i = 0, y_i = 1\} \big|$$ Then $$A \sim Bin(\alpha n, \rho), B \sim Bin((1-\alpha)n,\rho).$$ We know that $\|y\| = \alpha n - A + B$. So $$T_\rho(Maj, x) = \Pr[\|y\| \geq \frac{n}{2}] = \Pr[\alpha n - A + B \geq \frac{n}{2}] = \Pr[B - A \geq (\frac{1}{2} - \alpha)n]$$

\EPF

\BT
\textbf{(Berry-Esseen Theorem)} Let $X_1 , . . . . , X_n$ be independent random variables with $\textbf{E}[X_i]=0$ and $\textbf{Var}[X_i] = \sigma^{2}_{i}$, and assume $\sum_{i=1}^{n} \sigma^{2}_{i} = 1$. Let $\textbf{S}= \sum_{i=1}^{n} X_i$ and let $Z \sim N(0, 1)$ be a standard Gaussian. Then for all $u \in \mathbf{R}$, $$\big| \Pr[S \leq u] - \Pr[Z \leq u] \big| \leq c\gamma,$$ where $$\gamma = \sum\limits_{i=1}^{n} \|X_i\|^{3}_{3}$$ and c is a universal constant. 

\ET

\BCM
If $Z,Z^{'} \sim \textbf{N}(0,1)$. Then $$\big| \Pr[Z\leq t] - \Pr[Z^{'} \leq t^{'}] \big|$$ $$= \big| \Pr[Z\leq t] - \Pr[Z \leq t^{'}] \big| $$  $$= \Pr[min(t^{'},t) < Z \leq max(t^{'},t)] \leq  O\Big( \frac{|t-t^{'}|}{\sqrt{n}} \Big)$$

\ECM

\BCM
Let  $A = Bin(\alpha n, \rho)$, $B = Bin((1-\alpha)n, \rho)$. Let 

\begin{equation}
X  =\begin{cases}
1 , & \text{w.p $\rho$ }\\
0 , & \text{w.p $1-\rho$ }
\end{cases}
\end{equation}
and 
\begin{equation}
X^{'}  =\begin{cases}
-1 , & \text{w.p $\rho$ }\\
0 , & \text{w.p $1-\rho$ }
\end{cases}
\end{equation}

$X_1,...., X_{(1-\alpha)n} \sim X$ and $X_{(1-\alpha)n + 1}, ....., X_n \sim X^{'}$ and $$Y_i = \frac{X_i - E[X_i]}{\sqrt{n Var[X_i]}}$$

 Let $S = \sum_{i=1}^{n}Y_i$. Then $$\Pr\bigg[B-A \leq (\frac{1}{2} - \alpha)n\bigg] = \Pr\bigg[S \leq (\frac{1}{2} - \alpha)\sqrt{n} \frac{1-2\rho}{\sqrt{\rho(1-\rho)}}   \bigg]$$
\ECM

\BPF
We know that Binomial random variables can be written as sum of Bernoulli random variables. Hence $$\Pr\bigg[B-A \leq (\frac{1}{2} - \alpha)n\bigg] = \Pr\bigg[\sum_{i=1}^{n} X_i \leq (\frac{1}{2} - \alpha)n\bigg]$$ $$ = \Pr \Bigg[\frac{\sum_{i=1}^{n} X_i - \sum_{i=1}^{n} E[X_i]}{\sqrt{nVar[X]}} \leq \frac{(\frac{1}{2} - \alpha)n - \sum_{i=1}^{n} E[X_i]}{\sqrt{nVar[X]}}\Bigg]$$ $$ = \Pr\Bigg[S \leq \frac{(\frac{1}{2} - \alpha)n - \sum_{i=1}^{n} E[X_i]}{\sqrt{nVar[X]}}\Bigg] $$ Since $Var[X_1]= Var[X_2]= .... = Var[X_n] = Var[X] = \rho(1-\rho)$.  $E[X_1] = E[X_2] = ... = E[X_{(1-\alpha)n}] = -E[X_{(1-\alpha)n + 1}] = ... = -E[X_n] = \rho$, so $\sum_{i=1}^{n} E[X_1] = (\frac{1}{2} - \alpha)2\rho n$. Substituting these values the above formula simplifies to  $$  \Pr\bigg[S \leq (\frac{1}{2} - \alpha)\sqrt{n} \frac{1-2\rho}{\sqrt{\rho(1-\rho)}} \bigg] $$
\EPF

\BCM
Let $Z\sim N(0,1)$ and $S= \sum_{i=1}^{n}Y_i $ where $Y_1, Y_2,..., Y_n, X_1,...., X_n, X, X^{'}$ are same as described in Claim 1.8. Then for all $u\in \mathbf{R}$ $$\big| \Pr[S \leq u] - \Pr[Z \leq u] \big| \leq c \frac{1}{\sqrt{n}} \Big( \frac{\rho^2 + (1-\rho)^2}{\sqrt{\rho(1-\rho)}} \Big)$$ Where c is a constant.

\ECM

\BPF
Here we can use Theorem 1.6 because for all i $E[Y_i] = 0$ and $\sum_{i=1}^{n}Var[Y_i] = 1$ By applying Theorem 1.6 we get $$\big| \Pr[S \leq u] - \Pr[Z \leq u] \big| \leq c\gamma,$$ where $\gamma = \sum\limits_{i=1}^{n} \|Y_i\|^{3}_{3}$. Lets calculate $\gamma$ in this case and plug it in.  $$ \gamma = \sum_{i=1}^{n}E[|Y_i|^3]$$ On substituting $Y_i$ we get 
$$\gamma = \sum_{i=1}^{n}E\Bigg[ \frac{|X_i - E[X_i]|^3}{|nVar[X_i]|^{\frac{3}{2}}} \Bigg]$$
Since the value of $Var[X_i] = Var[X]$ for all $i$ we can bring $nVar[X]$ outside summation.
$$ \gamma= \frac{1}{(nVar[X])^{\frac{3}{2}}} \sum_{i=1}^{n} E[|X_i - E[X_i]|^3]$$
On calculating the expectation we get
$$ \gamma= \frac{1}{(nVar[X])^{\frac{3}{2}}} \sum_{i=1}^{n} (\rho(1-\rho)^3 + (1-\rho)\rho^3)$$
$$ \gamma= \frac{n\rho(1-\rho)(\rho^2 + (1-\rho)^2)}{(nVar[X])^{\frac{3}{2}}}   $$
Substituting the value of $Var[X]= \rho(1-\rho)$ we get 
$$\gamma =  \frac{1}{\sqrt{n}} \Big( \frac{\rho^2 + (1-\rho)^2}{\sqrt{\rho(1-\rho)}} \Big)$$

\EPF





\BCM
Let $\alpha - \epsilon \leq \beta \leq \alpha + \epsilon$, $A = Bin(\alpha n, \rho)$, $B = Bin((1-\alpha)n, \rho)$, $A^{'} = Bin(\beta n, \rho)$, $B^{'} = Bin((1-\beta) n, \rho)$. Then $$ \big| \Pr[B^{'}- A^{'}  \geq (\frac{1}{2} - \beta)n] - \Pr[B-A \geq (\frac{1}{2} - \alpha)n] \big| = O\Big(\frac{\epsilon}{\sqrt{\rho}}\Big).$$ When $\epsilon = \omega(\frac{1}{\sqrt{n}})$ .

\ECM

\BPF
$$\big| \Pr[B^{'}- A^{'}  \geq (\frac{1}{2} - \beta)n] - \Pr[B-A \geq (\frac{1}{2} - \alpha)n] \big|$$ $$= \big|\big(1- \Pr[B^{'}- A^{'}  \leq (\frac{1}{2} - \beta)n]\big) - \big(1 - \Pr[B-A \leq (\frac{1}{2} - \alpha)n]\big) \big|$$  $$= \big| \Pr[B^{'}- A^{'}  \leq (\frac{1}{2} - \beta)n] - \Pr[B-A \leq (\frac{1}{2} - \alpha)n] \big|$$

Using claim 1.8 we can write above equation as
 $$= \Bigg| \Pr\bigg[S \leq (\frac{1}{2} - \beta)\sqrt{n} \frac{1-2\rho}{\sqrt{\rho(1-\rho)}}   \bigg] - \Pr\bigg[S \leq (\frac{1}{2} - \alpha)\sqrt{n} \frac{1-2\rho}{\sqrt{\rho(1-\rho)}}   \bigg] \Bigg| $$   

We add and subtract CDF's of normal distribution in order to use Berry Esseen theorem.

$$ =  \Bigg| \Pr\bigg[S \leq (\frac{1}{2} - \beta)\sqrt{n} \frac{1-2\rho}{\sqrt{\rho(1-\rho)}} \bigg] - \Phi((\frac{1}{2} - \beta)\sqrt{n} \frac{1-2\rho}{\sqrt{\rho(1-\rho)}}) + \Phi((\frac{1}{2} - \beta)\sqrt{n} \frac{1-2\rho}{\sqrt{\rho(1-\rho)}})$$  $$- \Phi((\frac{1}{2} - \alpha)\sqrt{n} \frac{1-2\rho}{\sqrt{\rho(1-\rho)}}) + \Phi((\frac{1}{2} - \alpha)\sqrt{n} \frac{1-2\rho}{\sqrt{\rho(1-\rho)}})- \Pr\bigg[S \leq (\frac{1}{2} - \alpha)\sqrt{n} \frac{1-2\rho}{\sqrt{\rho(1-\rho)}}\bigg] \Bigg| $$  

Applying triangle inequality

$$\leq \Bigg| \Pr\bigg[S \leq (\frac{1}{2} - \beta)\sqrt{n} \frac{1-2\rho}{\sqrt{\rho(1-\rho)}}\bigg] - \Phi((\frac{1}{2} - \beta)\sqrt{n} \frac{1-2\rho}{\sqrt{\rho(1-\rho)}})\Bigg|$$ $$ + \Bigg|\Phi((\frac{1}{2} - \beta)\sqrt{n} \frac{1-2\rho}{\sqrt{\rho(1-\rho)}}) - \Phi((\frac{1}{2} - \alpha)\sqrt{n} \frac{1-2\rho}{\sqrt{\rho(1-\rho)}})\Bigg|$$ $$ + \Bigg|\Phi((\frac{1}{2} - \alpha)\sqrt{n} \frac{1-2\rho}{\sqrt{\rho(1-\rho)}})- \Pr\bigg[S \leq (\frac{1}{2} - \alpha)\sqrt{n} \frac{1-2\rho}{\sqrt{\rho(1-\rho)}}\bigg] \Bigg| $$ using claim 1.7 and Berry-Esseen theorem 

$$= O\Bigg(\frac{1}{\sqrt{n}} \Big( \frac{\rho^2 + (1-\rho)^2}{\sqrt{\rho(1-\rho)}} \Big)\Bigg) + O\Bigg(\epsilon \frac{1-2\rho}{\sqrt{\rho(1-\rho)}}\Bigg) + O\Bigg(\frac{1}{\sqrt{n}} \Big( \frac{\rho^2 + (1-\rho)^2}{\sqrt{\rho(1-\rho)}} \Big)\Bigg) $$

Since we are considering $\epsilon = \omega(\frac{1}{\sqrt{n}})$ and $\rho \leq \frac{1}{2}$ the above equation will simplify to
$$= O\Big(\frac{\epsilon}{\sqrt{\rho}}\Big)$$

%$$ = \big| \Pr[B^{'}- A^{'}  \leq (\frac{1}{2} - \beta)n] - \Phi((\frac{1}{2} - \beta)n) + \Phi((\frac{1}{2} - \beta)n)  - \Phi() - \Pr[B-A \leq (\frac{1}{2} - \alpha)n] \big|$$

\EPF


\BT
We can estimate $T_{\rho} \otimes Maj(X)$ to accuracy $\pm \gamma$ with $q= O(\frac{1}{\gamma^2 \rho}log\frac{1}{\delta})$ queries. 
\ET

\BPF 

\textbf{Algorithm}\\
Estimate the Hadamard weight of X by making  $O(\frac{1}{\gamma^2 \rho}log\frac{1}{\delta})$ queries and then calculate $T_{\rho} \otimes Maj(X)$ by using the formula in Claim 1.5. Instead of $\alpha$ use our estimate of hamming weight. 

\textbf{Analysis}\\
Let $\epsilon$ to be the error in the estimate of hamming weight and let $\rho$ be the noise parameter. Using Lemma 1.4 we have $\epsilon \leq \gamma \sqrt{\rho}$ if we make 
$$O(\frac{1}{\gamma^2 \rho}log\frac{1}{\delta})$$
queries. So $$\frac{\epsilon}{\sqrt{\rho}} \leq \gamma$$

From claim 1.10 we know that the equation below is satisified if the above equation is satisified 

$$ \big| \Pr[B^{'}- A^{'}  \geq (\frac{1}{2} - \beta)n] - \Pr[B-A \geq (\frac{1}{2} - \alpha)n] \big| \leq \gamma$$

If we try to estimate the hamming weight and then calculate $T_{\rho} \otimes Maj(X)$, From Claim 1.5 we know that the error in our estimate is
 
 $$ \big| \Pr[B^{'}- A^{'}  \geq (\frac{1}{2} - \beta)n] - \Pr[B-A \geq (\frac{1}{2} - \alpha)n] \big| $$
 
Hence our estimate of $T_{\rho} \otimes Maj(X)$ is accurate up to error $\pm \gamma$
\EPF


\subsection{Tribes Function}


In the case of Majority we were able to estimate $T_{\rho} \otimes Maj(X)$ well using the estimate of hamming weight.  Lets try to see if $T_{\rho} \otimes Tribes(X)$ can be estimated well using hamming weight. We will be considering the unbiased tribes function. An unbiased tribes function with n variables will be having $width \simeq \log n - \log \log n$. 

\BT
Even with a good estimate of hamming weight of X, we cannot estimate $T_{\rho} \otimes Tribes(X)$.

\ET
 
 To prove this we will be looking at two inputs $X_1, X_2$ with same hamming weight and show that $| T_{\rho} \otimes Tribes(X_1) - T_{\rho} \otimes Tribes(X_2)|$ large for a constant $\rho$.

\BCM
Let input $X_1= (11...10...00)(11...10...00)......(11...10...00)$. In $X_1$ each clause has half 1's and half 0's. $T_{\rho} \otimes Tribes(X_1) \leq O(\frac{1}{n^3})$, for noise parameter $\rho = \frac{1}{2^{10}}$.

\ECM

\BPF
From Definition 1.1 $$T_{\rho} \otimes Tribes(X_1) = \underset{Y_1 \sim N_{\rho}(X_1)}{\mathbb{E}}[Tribes(Y_1)]$$
and $Tribes(Y_1) \leq$ Number of clauses true in $Y_1$. Hence 
$${\mathbb{E}}[Tribes(Y_1)] \leq {\mathbb{E}}[\text{Number of clauses true in } Y_1]$$

The probability that a clause is true is 

$$\leq \Big(\frac{1}{2^{10}} \Big)^{\frac{\log n -\log \log n}{den}} $$

On using union bound 
$${\mathbb{E}}[\text{Number of clauses true in } Y_1] \leq \Big(\frac{1}{2^{10}} \Big)^{\frac{\log n -\log \log n}{den}} \frac{n}{\log n - \log \log n} $$

$$= \frac{(\log n)^5}{n^5} \frac{n}{\log n - \log \log n}$$
$$\leq O\Big(\frac{1}{n^3}\Big)$$

\EPF

\BCM
Let input $X_3 = (11...11)(11...11).....(11...11)(11...11).....(11...11)(11...11)$. Then $T_{\rho} \otimes Tribes(X_3) \geq \frac{1}{2}$, for $\rho \leq \frac{1}{2}$.
\ECM

\BPF
The function we are considering is unbiased so 
$$T_{\frac{1}{2}} \otimes Tribes(X_3) = \frac{1}{2}$$
This is because having $\rho = \frac{1}{2}$ is equivalent to generating a random n-bit string. For $\rho \leq \frac{1}{2}$ 
$$\underset{Y_3 \sim N_{\frac{1}{2}}(X_3)}{\mathbb{E}}[Tribes(Y_3)] \leq \underset{Y_3 \sim N_{\rho}(X_3)}{\mathbb{E}}[Tribes(Y_3)]$$
Since we are starting with all 1's the probability that the output is 1 is less if there is more noise. From the above two equations and definition 1.1 you can conclude that 
$$T_{\rho} \otimes Tribes(X_3) \geq \frac{1}{2}$$ for $\rho \leq \frac{1}{2}$.
 
\EPF

The above bound is very loose. But to prove our theorem that is enough.

\BCM
Let input $X_4 = (11...11)(11...11).....(11...11)$. $X_4$ is the first half of input $X_2$, it has $\frac{n}{2}$ bits. Then $\underset{Y_4 \sim N_{\rho}(X_4)}{\mathbb{E}}[\text{Atleast one clause is true in }Y_4] \geq \frac{1}{4}$. For $\rho \leq \frac{1}{2}$.

\ECM

\BPF
$X_3$ can be looked at as concatenation of two $X_4 {'} s$. So using union bound
$$\underset{Y_3 \sim N_{\rho}(X_3)}{\mathbb{E}}[Tribes(Y_3)] \leq 2\underset{Y_4 \sim N_{\rho}(X_4)}{\mathbb{E}}[\text{Atleast one clause is true in }Y_4]$$
Using Claim 1.14 and the above equation we can conclude that
$$\underset{Y_4 \sim N_{\rho}(X_4)}{\mathbb{E}}[\text{Atleast one clause is true in }Y_4] \geq \frac{1}{4}$$
 

\EPF




\BCM
Let input $X_2 = (11...11)(11...11).....(11...11)(00...00).....(00...00)(00...00)$. In $X_2$ you have first half of the clauses all 1's and the second half all 0's. Then $T_{\rho} \otimes Tribes(X_2) \geq \frac{1}{4}$ for $\rho \leq \frac{1}{2}$.

\ECM

\BPF
Since $X_2$ is just a concatenation of $X_4$ with 0's. We have the following inequality

$$T_{\rho} \otimes Tribes(X_2) = \underset{Y_2 \sim N_{\rho}(X_2)}{\mathbb{E}}[Tribes(Y_2)]  \geq \underset{Y_4 \sim N_{\rho}(X_4)}{\mathbb{E}}[\text{Atleast one clause is true in }Y_4]$$

And from Claim 1.15 we can conclude that 
$$T_{\rho} \otimes Tribes(X_2) \geq \frac{1}{4}$$
for $\rho \leq \frac{1}{2}$.
\EPF

\BT
Even with a good estimate of hamming weight of X, we cannot estimate $T_{\rho} \otimes Tribes(X)$.

\ET

\BPF
From Claim 1.13 and Claim 1.16. We have shown two inputs $X_1$ and $X_2$ both with same hamming weight($\frac{1}{2}$) but  $| T_{\rho} \otimes Tribes(X_1) - T_{\rho} \otimes Tribes(X_2)| \geq c$. Where c is a constant less than $\frac{1}{4}$. Hence we can conclude that even though we have the exact hamming weight of X its not enough to estimate $T_{\rho} \otimes Tribes(X)$ accurately.

\EPF



% hamming

\iffalse


\subsection{Biased Noise Operator}

\BD
$T^{+}_{\rho}(f,x)= \underset{y \sim N_{\rho}(x)}{\mathbb{E}}[f(y)]$
\ED
 
 
\noindent
For $T^{+}_{\rho}(f,x)$
\BE
 \item $\rho = 0$ : same as the previous case.
 \item $\rho = \frac{1}{2}$ : Not the same as previous case.
\EE 

\noindent
 Options we are left with
\BE
\item Abandon worst case.
\item Change notion of success.
 \item Restrict the problem set $\rho= \frac{1}{4}$.(We will be looking at this first)
\EE



 


\BCM
 Fix $X \in \{0,1\}^n$, let $\alpha = \frac{\|X\|}{n}$, let $A \sim Bin((1-\alpha)n, \frac{1}{4})$. Then $T_{\frac{1}{4}}^+ (f,x) = Pr[A \geq (\frac{1}{2} - \alpha)n]$.
\ECM
\BPF
By definition.
\EPF

\BT
\textbf{Kolmogorov-Rogozin 1959-61:} Let X be a random variable,  $\lambda>0$ be a constant, I be an interval and $Q(\lambda, X) = sup_{|I|=\lambda} \textbf{P}(X \in I)$. $S = X_1 + \cdots + X_n$ where $X_i$ are independent. Then $Q(\lambda, S) = O\Big(\frac{1}{\sqrt{\sum_{1}^{n}(1- Q(\lambda, X_i))}}\Big)$
\ET

\BCM
Let $\alpha$, $\beta$, $\rho$ be fixed. $A \sim Bin((1-\alpha)n, \rho)$, then $\big| Pr[A \geq (\frac{1}{2} - \beta)n] - Pr[A \geq (\frac{1}{2} - \alpha)n] \big| \leq O\Big(\frac{1}{\sqrt{n\rho(1-\alpha)}}\Big)$.
\ECM
\BPF
Let $X_i$ be a Bernoulli random variable with a success probability $\rho$, then $\forall \lambda > 0$, $Q(\lambda, X_i)= \rho$. Since $A = \Sigma_{i = 1}^{(1-\alpha)n} X_i$, from Theorem 1.5 we get $ \forall\lambda>0$, $Q(\lambda, A) = O\Big(\frac{1}{\sqrt{n\rho(1-\alpha)}}\Big)$.
\EPF


\BCM
Fix X, $\alpha$, $\epsilon > 0$. Let $A \sim Bin((1-\alpha)n, \frac{1}{4})$, $B \sim Bin((1-\beta)n, \frac{1}{4})$. Let $\alpha - \epsilon \leq \beta \leq \alpha + \epsilon$. Then $\big| Pr[B \geq (\frac{1}{2} - \beta)n] - Pr[A \geq (\frac{1}{2} - \alpha)n] \big| \leq \gamma$.
\ECM
\BPF
\\ \noindent
We directly can't compare $Pr[B \geq (\frac{1}{2} - \beta)n]$ and $Pr[A \geq (\frac{1}{2} - \alpha)n]$. So we bring in an intermediate quantity $Pr[A \geq (\frac{1}{2} - \beta)n]$. To upper bound $\big| Pr[B \geq (\frac{1}{2} - \beta)n] - Pr[A \geq (\frac{1}{2} - \alpha)n] \big|$ we upper bound $\big| Pr[A \geq (\frac{1}{2} - \beta)n] - Pr[A \geq (\frac{1}{2} - \alpha)n] \big|$ and $\big| Pr[B \geq (\frac{1}{2} - \beta)n] - Pr[A \geq (\frac{1}{2} - \beta)n] \big|$ separately and add the bounds. From Claim 1.6 we already have a bound on $\big| Pr[A \geq (\frac{1}{2} - \beta)n] - Pr[A \geq (\frac{1}{2} - \alpha)n] \big|$, so if could get a bound for $\big| Pr[B \geq (\frac{1}{2} - \beta)n] - Pr[A \geq (\frac{1}{2} - \beta)n] \big|$ we are done.

%\textbf{Case 1: $\beta = \alpha + \epsilon$} \\ \noindent
%Want to show $Pr[A \geq (\frac{1}{2} - \alpha- \epsilon)n] - Pr[A \geq (\frac{1}{2} - \alpha)n] \leq \gamma$. This is same as $Pr[(\frac{1}{2} - \alpha- %\epsilon)n \leq A < (\frac{1}{2} - \alpha)n] \leq \gamma$.

\EPF

%Let $$y_i = $$       
 
\fi 



%intersection. Then for any element $u\in \Omega_1$ there is an element $g_1 \in G$ such
%that $u^{g_1} = x$ and for any element $v\in \Omega_2$ there is an element $g_2\in G$ such
%that $v^{g_2}=v$. Thus the element $g_1g_2\in G$ maps $u$ to $v$, and so these orbits
%are equal. \EPF


\end{document}